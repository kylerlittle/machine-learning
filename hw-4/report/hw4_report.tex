\documentclass[12pt]{article}
\usepackage[margin=1in]{geometry}
\usepackage{amsmath}
\usepackage{graphicx}
\usepackage{stmaryrd}
\usepackage{float}

\author{Kyler Little\vspace{-0.6cm}}
\title{Homework \#4: Machine Learning\vspace{-0.3cm}}
\date{March 25, 2018\vspace{-0.7cm}}

\begin{document}
	\maketitle
	\section*{Problem \#1}
		Exercise 8.5 \\ 
		Show that the matrix Q described in the linear hard-margin SVM algorithm above is positive semi-definite (that is $u^T \text{Q}u \ge 0$ for any $u$).\\
		From the problem statement, $\text{Q}$ is defined to be:
		\begin{center}
			$\text{Q} = \left[ 
			\begin{array}{cc}
			0 & \boldsymbol{0}_d^T \\
			\boldsymbol{0}_d & I_d\\
			\end{array}
			\right]$
		\end{center}
		Let $u = \{u_0 \} \times x \in R^{d+1}$, where I augment an arbitrary element $u \in R$ to $x \in R^d$. This simplifies notation a little bit. Then:
		\begin{align*}
			u^T\text{Q}u &= [u_0 \ x_1 \cdots \ x_d] Q [u_0 \ x_1 \cdots \ x_d]^T\\
			&=[0 \ x_1 \cdots \ x_d] [u_0 \ x_1 \cdots \ x_d]^T\\
			&= ||x||^2 \ge 0
		\end{align*}
		Thus, Q is positive semi-definite.
	 
	\section*{Problem \#2}
	Exercise 8.11 \\
	(a) Show that the problem in (8.21) is a standard QP-problem:
	\begin{align*}
	\underset{\alpha \in R^{N}}{\text{minimize}} &\qquad \frac{1}{2}\alpha^T\text{Q}_D\alpha-1_{N}^{T}\alpha \\
	\text{subject to} & \qquad A_D\alpha \ge 0_{N+2} \\
	\end{align*}
	where $Q_D$ and $A_D$ ($D$ for the dual) are given by:\\
	\begin{center}
		$Q_D = \left[
		\begin{array}{ccc}
		y_1y_1x_1^Tx_1&\dots&y_1y_Nx_1^Tx_N \\
		y_2y_1x_2^Tx_1&\dots&y_2y_Nx_2^Tx_N \\
		\vdots & \vdots & \vdots \\
		y_Ny_1x_N^Tx_1&\dots&y_Ny_Nx_N^Tx_N \\
		\end{array}
		\right]$ and $A_D = \left[ 
		\begin{array}{c}
		y^T\\
		-y^T\\
		I_{N x N}\\
		\end{array} \right]$
	\end{center}
	Starting from the standard QP-problem, I will derive the original problem (8.21) to show equivalence. \\
	Firstly, it is obvious that $-\sum_{i=1}^{N}\alpha_i = [-1\ \cdots\ -1][\alpha_1 \ \cdots \alpha_N]^T$. Next, we will show equivalence of the first terms.
	\begin{align*}
		\sum_{i=1}^{N}\sum_{j=1}^{N}y_i y_j \alpha_i \alpha_j x_i^T x_j &= \alpha^T \text{Q}_D \alpha \\
		&= [\alpha_1 \ \cdots \ \alpha_N] \text{Q}_D [\alpha_1 \ \cdots \ \alpha_N]^T \\
		&= [\sum_{i=1}^{N}\alpha_iy_iy_1x_i^Tx_1 \ \cdots \ \sum_{i=1}^{N}\alpha_iy_iy_Nx_i^Tx_N] [\alpha_1 \ \cdots \ \alpha_N]^T \\
		&= \sum_{i=1}^{N}\sum_{j=1}^{N}y_i y_j \alpha_i \alpha_j x_i^T x_j
	\end{align*}
	Lastly, I'll show equivalence of the constraints.
	\begin{center}
		$A_D \alpha = \left[ 
		\begin{array}{c}
		y^T\\
		-y^T\\
		I_{N x N}\\
		\end{array} \right] 
		\left[ 
		\begin{array}{c}
		\alpha_1\\
		\vdots\\
		\alpha_N\\
		\end{array} \right] = \left[ 
		\begin{array}{c}
		\sum_{i=1}^{N}y_i\alpha_i\\
		-\sum_{i=1}^{N}y_i\alpha_i\\
		\alpha_1\\
		\vdots \\
		\alpha_N\\
		\end{array} \right] \ge 0$
	\end{center}
	This directly implies that $\alpha_i \ge 0, \ \forall i \in \{1,\dots,N\}$. And then only way for $\sum_{i=1}^{N}y_i\alpha_i \ge 0$ and $-\sum_{i=1}^{N}y_i\alpha_i \ge 0$ is if $\sum_{i=1}^{N}y_i\alpha_i = 0$. Thus, the problems are equivalent.
	\\(b) The matrix $Q_d$ of quadratic coefficients is $[Q_d]_{mn} = y_m y_n x^T_m x_n$.
	Show that $Q_d = X_s X^T_s$, where $X_s$ is the `signed data matrix',\\ 
	\begin{center}
			$X_s = \left[
			\begin{array}{c}
			y_1x_1^T\\
			y_2x_2^T\\
			\vdots\\
			y_Nx_N^T\\
			\end{array} \right]$
	\end{center}
	Hence, show that $Q_D$ is positive semi-definite. \\
	First, note that $X_s^T = [y_1x_1 \ \cdots \ y_Nx_N]$. From there, it is pretty easy to see that 
		\begin{center}
			$X_s X_s^T= \left[
			\begin{array}{c}
			y_1x_1^T\\
			y_2x_2^T\\
			\vdots\\
			y_Nx_N^T\\
			\end{array} \right] [y_1x_1 \ \cdots \ y_Nx_N] = \left[
			\begin{array}{ccc}
			y_1y_1x_1^Tx_1&\dots&y_1y_Nx_1^Tx_N \\
			y_2y_1x_2^Tx_1&\dots&y_2y_Nx_2^Tx_N \\
			\vdots & \vdots & \vdots \\
			y_Ny_1x_N^Tx_1&\dots&y_Ny_Nx_N^Tx_N \\
			\end{array}
			\right]$
		\end{center}
	using basic matrix multiplication.
	To show $Q_D$ is positive semidefinite, we first define arbitrary vector $y \in R^N$. Then, we have:
	\begin{align*}
		y^T \text{Q}_D y &= y^T X_sX_s^Ty \\
		&= (X_s^Ty)^T X_s^Ty\\
		&= ||X_s^Ty||^2 \ge 0
	\end{align*}
	Thus, Q is positive semi-definite.
	\section*{Problem \#3}
	Exercise 8.13 \\
	KKT complementary slackness gives that if $\alpha_n^* > 0$, then $(x_n , y_n)$ is on
	the boundary of the optimal fat-hyperplane and $y_n (w^{*T} x_n + b^{*} ) = 1$.
	Show that the reverse is not true. Namely, it is possible that $\alpha_n^* = 0$ and
	yet $(x_n , y_n)$ is on the boundary satisfying $y_n (w^{*T} x_n + b^{*} ) = 1$.
	[Hint: Consider a toy data set with two positive examples at (0, 0) and
	(1, 0), and one negative example at (0, 1).] \\
	Assuming we have that $y_n (w^{*T} x_n + b^{*} ) = 1$, we need to show that it's possible for $\alpha_n^* = 0$. We can do so by using the toy data set described in the hint. In that problem, the optimal hyperplane is $w = [0 \ -2]^T$ where $b=1$. For this case, the class labels would have to be $+1$ and $-1$. It's easy to see that the point (0, 1) is not a support vector. If we remove it, the optimal hyperplane wouldn't change. Thus, $\alpha_n^* = 0$ for that data point. All that's left to show is that $y_n (w^{*T} x_n + b^{*} ) = 1$ for all three data points. If this is the case, we have all data points lying on the boundary but one point exists with $\alpha_n^* = 0$. For (0, 0), we have $+1 ((0*0 + -2*0) + +1)=1$. For (1, 0), we have $-1 ((0*0 + -2*1) + 1)=1$. Lastly, for (1, 0), we have $-1 ((0*1 + -2*0) + 1)=1$. Thus, there exists a counterexample, so the statement cannot be true.
	\section*{Problem \#4}
	Problem 8.1 \\
	Consider a data set with two data points $x_{\pm} \in R^d$ having class $\pm1$ respectively. Manually solve (8.4) by explicitly minimizing $||w||^2$.
	subject to the two separation constraints.
	Compute the optimal (maximum margin) hyperplane ($b^* , w^* )$ and its margin.
	Compare with your solution to Exercise 8.1.
	
	\section*{Problem \#5}
	Problem 8.2 \\
	Consider a data set with three data points in $R^2$: \\
	\begin{center}
		$X = \left[
		\begin{array}{cc}
		0 & 0 \\
		0 &-1 \\
		-2 & 0 \\
		\end{array}
		\right]$ and $y = \left[ 
		\begin{array}{c}
		-1\\
		-1\\
		+1\\
		\end{array} \right]$
	\end{center}
	Manually solve (8.4) to get the optimal hyperplane ($b^* , w^* )$ and its margin.
	
	\section*{Problem \#6}
	Problem 8.4 \\
	Set up the dual problem for the toy data set in Exercise 8.2. Then, solve the dual problem and compute $\alpha^*$, the optimal Lagrange
	multipliers.
	
\end{document}